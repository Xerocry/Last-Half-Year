\documentclass[a4paper]{article}
\usepackage[utf8x]{inputenc}
\usepackage[T1,T2A]{fontenc}
\usepackage[russian]{babel}
\usepackage{hyperref}
\usepackage{indentfirst} % включить отступ у первого абзаца
\usepackage{listings}
\usepackage{color}
\usepackage{here}
\usepackage{listings}
\usepackage{graphicx}
%\usepackage{forest}

\lstset{ %
language=bash,                % choose the language of the code
basicstyle=\footnotesize,       % the size of the fonts that are used for the code
numbers=left,                   % where to put the line-numbers
numberstyle=\footnotesize,      % the size of the fonts that are used for the line-numbers
stepnumber=1,                   % the step between two line-numbers. If it is 1 each line will be numbered
numbersep=5pt,                  % how far the line-numbers are from the code
backgroundcolor=\color{white},  % choose the background color. You must add \usepackage{color}
showspaces=false,               % show spaces adding particular underscores
showstringspaces=false,         % underline spaces within strings
showtabs=false,                 % show tabs within strings adding particular underscores
frame=single,           % adds a frame around the code
tabsize=2,          % sets default tabsize to 2 spaces
captionpos=b,           % sets the caption-position to bottom
breaklines=true,        % sets automatic line breaking
breakatwhitespace=false,    % sets if automatic breaks should only happen at whitespace
escapeinside={\%*}{*)},          % if you want to add a comment within your code
postbreak=\raisebox{0ex}[0ex][0ex]{\ensuremath{\color{red}\hookrightarrow\space}}
}

\usepackage[left=2.5cm, top=2cm, right=2cm, bottom=2cm, nohead]{geometry}

\makeatletter
\newenvironment{btHighlight}[1][]
{\begingroup\tikzset{bt@Highlight@par/.style={#1}}\begin{lrbox}{\@tempboxa}}
	{\end{lrbox}\bt@HL@box[bt@Highlight@par]{\@tempboxa}\endgroup}

\newcommand\btHL[1][]{%
	\begin{btHighlight}[#1]\bgroup\aftergroup\bt@HL@endenv%
	}
	\def\bt@HL@endenv{%
	\end{btHighlight}%   
	\egroup
}
\newcommand{\bt@HL@box}[2][]{%
	\tikz[#1]{%
		\pgfpathrectangle{\pgfpoint{1pt}{0pt}}{\pgfpoint{\wd #2}{\ht #2}}%
		\pgfusepath{use as bounding box}%
		\node[anchor=base west, fill=orange!30,outer sep=0pt,inner xsep=1pt, inner ysep=0pt, rounded corners=3pt, minimum height=\ht\strutbox+1pt,#1]{\raisebox{1pt}{\strut}\strut\usebox{#2}};
	}%
}
\makeatother

\lstdefinestyle{SQL}{
	language={SQL},basicstyle=\ttfamily, 
	moredelim=**[is][\btHL]{`}{`},
	moredelim=**[is][{\btHL[fill=green!30,draw=red,dashed,thin]}]{@}{@},
}

%\definecolor{javared}{rgb}{0.6,0,0} % for strings
%\definecolor{javagreen}{rgb}{0.25,0.5,0.35} % comments
%\definecolor{javapurple}{rgb}{0.5,0,0.35} % keywords
%\definecolor{javadocblue}{rgb}{0.25,0.35,0.75} % javadoc
%
%\lstset{language=Java,
%	basicstyle=\ttfamily,
%	keywordstyle=\color{javapurple}\bfseries,
%	stringstyle=\color{javared},
%	commentstyle=\color{javagreen},
%	morecomment=[s][\color{javadocblue}]{/**}{*/},
%	numbers=left,
%	numberstyle=\tiny\color{black},
%	stepnumber=2,
%	numbersep=10pt,
%	tabsize=4,
%	showspaces=false,
%	showstringspaces=false}



\begin{document}
\begin{titlepage} % начало титульной страницы

\begin{center} % включить выравнивание по центру

\large Санкт-Петербургский Политехнический Университет Петра Великого\\
\large Институт компьютерных наук и технологий \\
\large Кафедра компьютерных систем и программных технологий\\[6cm]
% название института, затем отступ 4,5см

\huge Базы данных\\[0.5cm] % название работы, затем отступ 0,6см
\large Отчет по лабораторной работе №2\\[0.1cm]
% тема работы, затем отступ 3,7см
\end{center}

\begin{flushright}
\begin{minipage}{0.5\textwidth}
\begin{flushright}
\textbf{Работу выполнил:}

Раскин Андрей

{Группа:} 43501/3\\


\textbf{Преподаватель:} 

Мяснов А.В.
\end{flushright}
\end{minipage} % конец врезки
\end{flushright} % конец выравнивания по левому краю

\vfill % заполнить всё доступное ниже пространство

\begin{center}

\large Санкт-Петербург\\
\large \the\year % вывести дату

\end{center} % закончить выравнивание по центру

\thispagestyle{empty} % не нумеровать страницу
\end{titlepage} % конец титульной страницы

\vfill % заполнить всё доступное ниже пространство

\section{Цель работы}
Получить практические навыки работы с БД путём создания собственного генератора тестовых данных на языке Java.

\section{Ход выполнения работы}
\subsection{Окружение}
При разработке использовался язык Java 8.
Для описания сущностей базы данных, как объектов языка использовалась технология JPA. JPA (Java Persistence API) это спецификация Java EE и Java SE, описывающая систему управления сохранением java объектов в таблицы реляционных баз данных в удобном виде. Сама Java не содержит реализации JPA, однако есть существует много реализаций данной спецификации от разных компаний (открытых и нет). Это не единственный способ сохранения java объектов в базы данных (ORM систем), но один из самых популярных в Java мире.
Для сборки проекта используется Gradle - система автоматической сборки, построенная на принципах Apache Ant и Apache Maven, но предоставляющая DSL на языке Groovy вместо традиционной XML-образной формы представления конфигурации проекта.

Для PostgreSQL была создана база данных clinic\_db, пользователь использовался стандартный - \textbf{postgres}. 
Для подключения к базе данных необходимо указать хост, порт, имя базы данных и логин/пароль в специальном файле настроек для Ebean.

\subsection{Создание генератора}
Заполнение объекта в Java случайными данными на первый взгляд выглядит достаточно просто. Но при более близком рассмотрении, если например модель БД включает множество связанных классов сложность возрастает. Для генерации случайных данных использовался EnhancedRandom API. Это библиотека, предоставляющая методы для настраеваемой генерации различных типов даных.
Часть данных генерируется путём взятия случайно строчки  из файла. Такой метод используется в тех случаях, когда полноценная случайность не важна: имена, названия лекарств и тп.
\lstinputlisting[]{scripts/GenFile.java}
В остальных случаях используется генерирование с помощью EnhancedRandom API.
\lstinputlisting[]{scripts/GenRandom.java}
Данные генерируются для всех таблиц пропорционально их собственному коэффициенту. Например, при генерации в таблице болезней $N$ строк, в таблице лекарств будет сгенерировано $2N$ строк, в таблице назначений будет сгенерировано $4N$ строк и так далее для каждой таблице. Такой метод генерации сохраняет логику базы данных, ведь количество назначений в карточках обычно больше количества лекарств, а количество лекарств обычно больше количества болезней.

Кроме пропорционального заполнения, генератор обеспечивает логическую целостность данных в отдельных таблицах. Это ознчачает, например, что таблицы показаний и противопоказаний не содержат одинаковых пар лекарство + болезнь; таблица несовместимости лекарств не содержит повторяющихся значений; цены, даты и количества лекарств содержатся в определенных границах и др.
\lstinputlisting[]{scripts/FullGen.java}

\section{Выводы}
В данной работы было проведено знакомство с библиотекой Enchanced Random для Java, позволяющей наполнять объекты случайными данными. 
Написание собственного генератора намного более гибкое решение, чем добавление данных вручную. Это обусловлено тем, что при тестировании нас обычно не волнуют точные значения имен, цен и прочих параметров, в то время как пропорции данных между таблицами, контроль повторных значений и неявные зависимости между таблицами важны при тестировании.
\end{document}