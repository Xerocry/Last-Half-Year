\documentclass[14pt,a4paper,report]{article}
\usepackage[a4paper, mag=1000, left=2.5cm, right=1cm, top=2cm, bottom=2cm, headsep=0.7cm, footskip=1cm]{geometry}
\usepackage[utf8]{inputenc}
\usepackage[english,russian]{babel}
\usepackage{indentfirst}
\usepackage[dvipsnames]{xcolor}
\usepackage[colorlinks]{hyperref}
\usepackage{listings} 
\usepackage{fancyhdr}
\usepackage{caption}
\usepackage{graphicx}
\usepackage{csquotes}
\hypersetup{
	colorlinks = true,
	linkcolor  = black
}

\usepackage{titlesec}
\titleformat{\chapter}
{\Large\bfseries} % format
{}                % label
{0pt}             % sep
{\huge}           % before-code


\DeclareCaptionFont{white}{\color{white}} 

% Listing description
\usepackage{listings} 
\DeclareCaptionFormat{listing}{\colorbox{gray}{\parbox{\textwidth}{#1#2#3}}}
\captionsetup[lstlisting]{format=listing,labelfont=white,textfont=white}
\lstset{ 
	% Listing settings
	inputencoding = utf8,			
	extendedchars = \true, 
	keepspaces = true, 			  	 % Поддержка кириллицы и пробелов в комментариях
	language = C,            	 	 % Язык программирования (для подсветки)
	basicstyle = \small\sffamily, 	 % Размер и начертание шрифта для подсветки кода
	numbers = left,               	 % Где поставить нумерацию строк (слева\справа)
	numberstyle = \tiny,          	 % Размер шрифта для номеров строк
	stepnumber = 1,               	 % Размер шага между двумя номерами строк
	numbersep = 5pt,              	 % Как далеко отстоят номера строк от подсвечиваемого кода
	backgroundcolor = \color{white}, % Цвет фона подсветки - используем \usepackage{color}
	showspaces = false,           	 % Показывать или нет пробелы специальными отступами
	showstringspaces = false,    	 % Показывать или нет пробелы в строках
	showtabs = false,           	 % Показывать или нет табуляцию в строках
	frame = single,              	 % Рисовать рамку вокруг кода
	tabsize = 2,                  	 % Размер табуляции по умолчанию равен 2 пробелам
	captionpos = t,             	 % Позиция заголовка вверху [t] или внизу [b] 
	breaklines = true,           	 % Автоматически переносить строки (да\нет)
	breakatwhitespace = false,   	 % Переносить строки только если есть пробел
	escapeinside = {\%*}{*)}      	 % Если нужно добавить комментарии в коде
}



\begin{document}
\begin{titlepage} % начало титульной страницы

\begin{center} % включить выравнивание по центру

\large Санкт-Петербургский Политехнический Университет Петра Великого\\
\large Институт компьютерных наук и технологий \\
\large Кафедра компьютерных систем и программных технологий\\[6cm]
% название института, затем отступ 4,5см

\huge Защита информации\\[0.5cm] % название работы, затем отступ 0,6см
\large Реферат\\[0.1cm]
\large Редакции федерального закона о защите информации с 2012 года\\[5cm]
% тема работы, затем отступ 3,7см
\end{center}

\begin{flushright}
\begin{minipage}{0.5\textwidth}
\begin{flushright}
\textbf{Работу выполнил:}

Раскин Андрей

{Группа:} 43501/3\\


\textbf{Преподаватель:} 

Новопашенный Андрей Гелиевич 
\end{flushright}
\end{minipage} % конец врезки
\end{flushright} % конец выравнивания по левому краю

\vfill % заполнить всё доступное ниже пространство

\begin{center}

\large Санкт-Петербург\\
\large \the\year % вывести дату

\end{center} % закончить выравнивание по центру

\thispagestyle{empty} % не нумеровать страницу
\end{titlepage} % конец титульной страницы

\vfill % заполнить всё доступное ниже пространство

\tableofcontents

\newpage

\section{ФЗ от 28.07.2012 N 139 - О внесении изменений в Федеральный закон «О защите детей от информации, причиняющей вред их здоровью и развитию»}

\subsection{Изменения в статье 2 - «Основные понятия, используемые в настоящем Федеральном законе»}
Добавлены ранее отсутствующие понятия:
\begin{enumerate}
	\item Сайт в сети Интернет;
	\item Страница сайта в сети Интернет;
	\item Доменное имя;
	\item Сетевой адрес;
	\item Владелец сайта в сети Интернет;
	\item Провайдер хостинга.
\end{enumerate}

Эти понятия соответствуют времени их добавления и не являются полными. Такие понятия, как Доменное имя, Сетевой адрес и тп в будущем получили более широкое и техническое определение, более полно описывающие их.

\subsection{Дополнение статьёй 15.1 - «Единый реестр доменных имен, указателей страниц сайтов в сети «Интернет»}
Несмотря на то, что в названии закона основной упор ставится на изменения, которые он вносит в Федеральный закон «О защите детей от информации, причиняющей вред их здоровью и развитию», изменения в нем носят технический характер и сводятся к сведению к единообразию ряда терминов, корректировки терминов и процедур по маркировке продукции.\\

Данный закон ограничивает право на доступ и распространение информации и препятствует реализации права на судебную защиту. Закон противоречит нескольким статьям Конституции РФ, а также Европейской конвенции по правам человека. Введенные законом поправки и новые нормы должны быть отменены

\section{ФЗ от 05.04.2013 N 50 - «О внесении изменений в отдельные законодательные акты Российской Федерации в части ограничения распространения информации о несовершеннолетних, пострадавших в результате противоправных действий (бездействия)»}

В данном законе дополняется статья 15.1 «Основные понятия, используемые в настоящем Федеральном законе».
Включается новый пункт в список причин добавления в реестр запрещённых ресурсов:
\begin{displayquote}
	информации о несовершеннолетнем, пострадавшем в результате противоправных действий (бездействия), распространение которой запрещено федеральными законами
\end{displayquote}

\section{ФЗ от 07.06.2013 N 112-ФЗ «О внесении изменений в ФЗ «Об информации, информационных технологиях и о защите информации»}

\subsection{Изменения в статье 2 - «Основные понятия, используемые в настоящем Федеральном законе»}
В определении "Сайт в сети интернет" слова "через сеть "Интернет" заменятся словами "посредством информационно-телекоммуникационной сети "Интернет". \\

Также добавилось определение системы идентификации:

\begin{displayquote}
	Единая система идентификации и аутентификации - федеральная государственная информационная система, порядок использования которой устанавливается Правительством Российской Федерации и которая обеспечивает в случаях, предусмотренных законодательством Российской Федерации, санкционированный доступ к информации, содержащейся в информационных системах.
\end{displayquote}

\subsection{Изменения в статье 7 - «Общедоступная информация»}
Статья дополняется несколькими частями:
\begin{enumerate}
	\item Часть 4 - Об общедоступной информации в форме открытых данных.
	\item Часть 5 - Об информации, являющейся государственной тайной. Доступ к такой информации должен быть прекращён по требованию органа наделенного полномочиями по распоряжению такими сведениями.
	\item Часть 6 - Об информациир, нарущающей права обладателей информации. Доступ к ней должен быть ограничен по решению суда.
\end{enumerate}

\subsection{Изменения в статье 7 - «Государственные информационные системы»}
Добавляется часть 4, говорящая о том, что при использовании корпорациями общедоступной информации для её обработки и использования, информация об этом и результат должен быть также опубликован в сети Интернет.

\section{ФЗ от 02.07.2013 N 187-ФЗ - «О внесении изменений в отдельные законодательные акты Российской Федерации по вопросам защиты интеллектуальных прав в информационно-телекоммуникационных сетях»}

\subsection{Изменения в статье 1 - «Сфера действия настоящего Федерального закона»}
Положения ФЗ не распространяются на отношения, возникающие при правовой охране результатов интеллектуальной деятельности и приравненных к ним средств индивидуализации, за исключением случаев, предусмотренных настоящим Федеральным законом.

\subsection{Дополнение статьей 15.2 - «Порядок ограничения доступа к информации, распространяемой с нарушением исключительных прав на фильмы, в том числе кинофильмы, телефильмы»}

С момента вступления закона в силу появились новые требования к провайдерам и владельцам сайтов. 
\begin{enumerate}

	\item В течение суток с момента получения от оператора реестра уведомления о включении сайта в реестр провайдер хостинга должен проинформировать владельца сайта о необходимости незамедлительного удаления интернет-страницы, содержащей запрещенную информацию.
	
	При этом следует отметить, что от владельца сайта необходимо удаление именно интернет-страницы, а не запрещенной информации. Следовательно, вне зависимости от количества запрещенного контента удалению подлежит вся страница.
	
	\item В течение суток с момента получения от провайдера хостинга уведомления о включении доменного имени или указателя страницы сайта в реестр владелец обязан удалить страницу. В противном случае доступ ко всему сайту будет ограничен.
	
	\item В случае отказа или бездействия владельца сайта в течение суток провайдер обязан ограничить доступ к сайту.
\end{enumerate}

В данных требованиях присутствует очевидное противоречие: доступ будет ограничен не к запрещенной информации, а ко всему сайту или даже сетевому адресу. При этом не определена процедура уведомления провайдера со стороны оператора реестра; кроме того, в связи с техническими особенностями администрирования сайтов в сети Интернет многие владельцы сайтов не могут получить информацию в такой короткий промежуток времени, поэтому отведенный срок может рассматриваться как дискриминационный.\\

При этом в законе прописано право владельца сайта обжаловать решение о включении в реестр в течение трех месяцев.


\subsection{Изменения в статье 17 - «Ответственность за правонарушения в сфере информации, информационных технологий и защиты информации»}

Была добавлена часть 4:

\begin{displayquote}
	\emph{Провайдер хостинга и владелец сайта в сети "Интернет" не несут ответственность перед правообладателем и перед пользователем за ограничение доступа к информации и (или) ограничение ее распространения в соответствии с требованиями настоящего Федерального закона. [5]}
\end{displayquote}


\section{ФЗ от 28.12.2013 N 396-ФЗ - «О внесении изменений в отдельные законодательные акты Российской Федерации»}

\subsection{Изменения в статье 15.1 - «Единый реестр доменных имен, содержащие информацию, распространение которой в Российской Федерации запрещено»}

Была добавлена часть 2:

\begin{displayquote}
	\emph{Государственные информационные системы создаются и эксплуатируются с учетом требований, предусмотренных законодательством Российской Федерации о контрактной системе в сфере закупок товаров, работ, услуг для обеспечения государственных и муниципальных нужд. [6]}
\end{displayquote}

\section{ФЗ от 28.12.2013 N 398-ФЗ - «О внесении изменений в Федеральный закон "Об информации, информационных технологиях и о защите информации»}

\subsection{Изменения в статье 15.1 - «Единый реестр доменных имен, содержащие информацию, распространение которой в Российской Федерации запрещено»}

Теперь, порядок ограничения доступа к сайтам в сети «Интернет«, предусмотренный статьей, не применяется к информации, порядок ограничения доступа к которой предусмотрен статьей 15.3 настоящего Федерального закона.

\subsection{Дополнение статьей 15.3 - «Порядок ограничения доступа к информации, распространяемой с нарушением закона»}

Об организации работы по рассмотрению уведомлений о распространении в информационно-телекоммуникационных сетях, в том числе в сети Интернет, информации, содержащей призывы к \begin{itemize}
	\item Осуществлению экстремистской деятельности;
	\item Массовым беспорядкам;
	\item Участию в публичных мероприятиях, проводимых с нарушением установленного порядка.
\end{itemize}
и направлению требований о принятии мер по ограничению доступа к информационным ресурсам, распространяющим такую информацию.\\

Доступ к ресурсу, содержащему данную информации осуществляется незамедлительно, без контакта с владельцем ресурса, однако доступ к ресурсу может быть восстановлен через федеральные органы исполнительной власти [7].


\section{ФЗ от 05.05.2014 N 97-ФЗ - «О внесении изменений в Федеральный закон «Об информации, информационных технологиях и о защите информации»»}

\subsection{Дополнение статьей 10.1 - «Обязанности организатора распространения информации в сети "Интернет"»}


Добавлено определение:
\begin{displayquote}
	 «Организатор распространения информации» — это любое «лицо, осуществляющее деятельность по обеспечению функционирования информационных систем и (или) программ для электронных вычислительных машин, которые предназначены и (или) используются для приема, передачи, доставки и (или) обработки электронных сообщений пользователей сети „Интернет“»
\end{displayquote}

Если через сайт можно передать, доставить или обработать любое электронное сообщение, то владелец сайта или оператор информационной системы, эксплуатирующий ее по соглашению с владельцем, являются организаторами распространения информации. 

Обязанности:
\begin{enumerate}
    \item Обязан уведомлять Роскомнадзор о начале своей деятельности.
    
	\item Хранить на территории Российской Федерации информацию о фактах приема, передачи, доставки и (или) обработки голосовой информации, письменного текста, изображений, звуков или иных электронных сообщений пользователей сети «Интернет» и информацию об этих пользователях в течение шести месяцев с момента окончания осуществления таких действий, а также предоставлять указанную информацию уполномоченным государственным органам, осуществляющим оперативно-разыскную деятельность или обеспечение безопасности Российской Федерации, в случаях, установленных федеральными законами.
	
	\item Обязан обеспечивать реализацию требований к оборудованию и программно-техническим средствам.
	
	\item Организатором распространения информации может быть признан владелец любого сайта или страницы, где возможна обратная связь.
	
\end{enumerate}

\subsection{Дополнение статьей 10.2 - «Особенности распространения блогером общедоступной информации»}

В соответствии с этой статьей, владельцы сайтов, которые зарегистрированы в качестве сетевых изданий, не являются блогерами, то указания на то, что они не являются организаторами распространения информации, в новой редакции закона нет, а значит, они (СМИ в интернете) должны выполнять все обязанности, предусмотренные частями 2-4 ст.10.1. То есть хранить на территории России персональные данные россиян, уведомить о своей деятельности Роскомнадзор и поставить средства СОРМ.

В рамках установленных полномочий Роскомнадзор ведет реестр сайтов и страниц сайтов в сети «Интернет», на которых размещается общедоступная информация и доступ к которым в течение суток составляет более трех тысяч пользователей сети «Интернет», а также реестр организаторов распространения информации в сети «Интернет».

Обязанности для блогеров из реестра:

\begin{enumerate}
\item Идентификация. Блогеры должны будут размещать свои фамилию и инициалы + электронный адрес.
\item Общественно значимые сведения. Всем пользователям интернета с трехтысячной ежедневной аудиторией запрещается скрывать «общественно значимые сведения», равно как и фальсифицировать их.
\item Достоверность. Проверять достоверность публикуемой информации и "незамедлительно" удалять недостоверные сведения (не оговаривается, насколько незамедлительно).
\item Выборы и референдумы. Соблюдать правила публикации материалов о выборах и референдумах, то есть не публиковать данные опросов общественного мнения за пять дней до голосования.
\item Соблюдение законодательства. Следить за соблюдением законодательства, не распространять экстремистскую информацию, не раскрывать гостайну, не публиковать оскорбительные высказывания, а также избегать матерных выражений. (Наказания за эти действия уже прописаны в других законах).
\item Запрет на материалы. Не использовать "материалы, пропагандирующие порнографию, культ насилия и жестокости, а также материалы, содержащие нецензурную брань". (Наказания за эти действия уже прописаны в других законах).

\item Частная жизнь. Не допускать распространения информации о частной жизни гражданина с нарушением гражданского законодательства (и это уже прописано в других законах).

\item Следить за публикациями и комментариями на вашей странице от других пользователей. Следить за тем, что пишет сам блогер, и за другими пользователями, которые могут написать что-то на их странице (отвечать не только за содержание своих постов, но и за любой комментарий на своей странице).
\end{enumerate}

\section{ФЗ от 21.07.2014 N 222-ФЗ «О внесении изменений в Федеральный закон «О государственном регулировании деятельности по организации и проведению азартных игр»}

За нарушение данных правовых норм законодательство предусматривает уголовную и административную ответственность.\\ 

Согласно поправкам, лицо несёт уголовную или административную ответственность за факт организации азартной игры в Интернете, вне зависимости от размера полученного дохода4. Кроме того, распространение рекламы онлайн-казино также влечёт ответственность для владельца сайта, предусмотренную вышеупомянутой ст. 15.1 Федерального закона «Об информации».\\ 
Такая ответственность наступает в силу ч. 2 статьи 27 Федерального закона «О рекламе», вводящую запрет на публикацию рекламы основанных на риске игр и пари в сети Интернет.


\section{ФЗ от 21.07.2014 N 242-ФЗ - «О внесении изменений в отдельные законодательные акты Российской Федерации в части уточнения порядка обработки персональных данных в информационно-телекоммуникационных сетях»}

Добавлена статья 15.5 «Порядок ограничения доступа к информации, обрабатываемой с нарушением законодательства РФ в области персональных данных».\\
В ней говорится о "Реестре нарушителей прав субъектов персональных данных".


\section{ФЗ от 24.11.2014 N 364-ФЗ «О внесении изменений в Федеральный закон «б информации, информационных технологиях и о защите информации»}

\subsection{Изменения в статье 15.2 - «Порядок ограничения доступа к информации, распространяемой с нарушением авторских и (или) смежных прав»}

Добавлена информация о действиях авторских и (или) смежных прав.

\subsection{Дополнение статьей 15.6 «Порядок ограничения доступа к сайтам в сети "Интернет", на которых неоднократно и неправомерно размещалась информация, содержащая объекты авторских и (или) смежных прав»}

Речь идет о нарушениях авторских и смежных прав путем размещения соответствующей информации в Интернете. Помимо того, что каждый правообладатель по-прежнему имеет право требовать у оператора связи ограничить доступ к незаконно размещенной информации по судебному акту, теперь такое требование можно заявить владельцу сайта и до суда. Для этого необходимо будет направить владельцу сайта заявление о нарушении авторских или смежных прав.\\

Законом также предусмотрена возможность постоянного ограничения доступа к сайтам, на которых неоднократно и неправомерно размещалась соответствующая информация.

\section{ФЗ от 29.06.2015 N 188-ФЗ «О внесении изменений в Федеральный закон "Об информации, информационных технологиях и о защите информации»»}

\subsection{Дополнение статьей 12.1. - «Особенности государственного регулирования в сфере использования российских программ для электронных вычислительных машин и баз данных»}

В нём говорится о создании единого реестра российских программ для электронных вычислительных машин и баз данных в целях расширения использования российского ПО и подтверждения его российского происхождения, оказания государственной поддержки правообладателям ПО.

Отечественными будут считаться те программные продукты, сведения о которых внесены в соответствующий реестр. Решения о включении и об исключении из реестра будет принимать Минкомсвязь России на основании заключений экспертного совета, состоящего из представителей государственных заказчиков и российской ИТ-отрасли.

Для вхождение нужно соблюдение условий:
\begin{enumerate}
	\item Исключительные права	
	\item Оборотоспособность
	\item Годовая сумма выплат	
	\item Доступность сведений о программе	
	\item Безопасность	
	\item Лицензия	
\end{enumerate}

\section{ФЗ от 13.07.2015 N 263-ФЗ «О внесении изменений в отдельные законодательные акты Российской Федерации в части отмены ограничений на использование электронных документов при взаимодействии физических и юридических лиц с органами государственной власти и органами местного самоуправления»}

Добавляется статья 11.1. Обмен информацией в форме электронных документов при осуществлении полномочий органов государственной власти и органов местного самоуправления.

Органы государственной власти, органы местного самоуправления обязаны предоставлять по выбору граждан и организаций информацию в форме электронных документов.\\

Также закон распространяет эти требования не только на органы государственной власти, но и на «организации, осуществляющие в соответствии с федеральными законами отдельные публичные полномочия». При этом в законодательстве отсутствует перечень этих «публичных полномочий», или хотя бы намек на то, какие организации, по мнению законодателя, их исполняют. Это означает, что точно сказать, на какие именно организации распространяются данные требования, практически невозможно, и в случае возникновения споров придется выяснять отношения в суде. 


\section{ФЗ от 13.07.2015 N 264-ФЗ «О внесении изменений в Федеральный закон «Об информации, информационных технологиях и о защите информации»»}

\subsection{Изменения в статье 2 «Основные понятия, используемые в настоящем Федеральном законе»}

Было добавлено определение поисковой системы:

\begin{displayquote}
	\emph{Поисковая система - информационная система, осуществляющая по запросу пользователя поиск в сети "Интернет" информации определенного содержания и предоставляющая пользователю сведения об указателе страницы сайта в сети "Интернет" для доступа к запрашиваемой информации, расположенной на сайтах в сети "Интернет", принадлежащих иным лицам, за исключением информационных систем, используемых для осуществления государственных и муниципальных функций, оказания государственных и муниципальных услуг, а также для осуществления иных публичных полномочий, установленных федеральными законами. [15]}
\end{displayquote}

\subsection{Дополнение статьей 12.1 10.3 - «Обязанности оператора поисковой системы»}

В статье описываются обязанности оператора поисковой системы. \\
Каждый гражданин может обратиться с требованиями прекратить выдачу ссылок на страницу сайта с информацией о заявителе, которая нарушает законодательство. Заявитель вправе обратиться в суд, в случае отказа оператора [15].


\section{ФЗ от 23.06.2016 N 208-ФЗ «О внесении изменений в Федеральный закон «Об информации, информационных технологиях и о защите информации»}

В нём добавлена статья 10.4. Особенности распространения информации новостным агрегатором.\\
В частности, под новостным агрегатором понимаются 
\begin{displayquote}
	Программы для ЭВМ, сайт и (или) страницы сайта в сети "Интернет", которые используются для обработки и распространения новостной информации в сети "Интернет" на государственном языке РФ, государственных языках республик в составе РФ или иных языках народов РФ, на которых может распространяться реклама, направленная на привлечение внимания потребителей, находящихся на территории РФ, и доступ к которым в течение суток составляет более одного миллиона пользователей сети "Интернет".
\end{displayquote}
При этом к новостным агрегаторам не относятся зарегистрированные сетевые издания.\\
Согласно документу, владельцы новостных агрегаторов обязаны проверять достоверность общественно значимой информации до ее распространения и удалять ее по предписанию Роскомнадзора.\\
При этом они не будут нести ответственность за распространение недостоверной информации в случае, если она является дословным цитированием СМИ. А владельцем новостного агрегатора может быть только российское юрлицо или гражданин РФ.

\section{ФЗ от 06.07.2016 N 374-ФЗ «О внесении изменений в Федеральный закон «О противодействии терроризму»»}

в статью 10.1 внесены изменения, согласно которым Организатор распространения информации в сети «Интернет» обязан хранить на территории Российской Федерации:
\begin{enumerate}


	\item Информацию о фактах приема, передачи, доставки и (или) обработки голосовой информации, письменного текста, изображений, звуков, видео- или иных электронных сообщений пользователей сети «Интернет» и информацию об этих пользователях в течение одного года с момента окончания осуществления таких действий;

	\item Текстовые сообщения пользователей сети «Интернет», голосовую информацию, изображения, звуки, видео-, иные электронные сообщения пользователей сети «Интернет» до шести месяцев с момента окончания их приема, передачи, доставки и (или) обработки. Порядок, сроки и объем хранения указанной в настоящем подпункте информации устанавливаются Правительством Российской Федерации" (вступает в силу с 01.07.2018).
\end{enumerate}

Наряду с этим, согласно п. 3.1. Организатор распространения информации в сети «Интернет» обязан предоставлять указанную информацию уполномоченным государственным органам, осуществляющим оперативно - разыскную деятельность или обеспечение безопасности Российской Федерации, в случаях, установленных федеральными законами».


\section{ФЗ от 19.12.2016 N 442-ФЗ «О внесении изменения в статью 15.1 Федерального закона «Об информации, информационных технологиях и о защите информации»»}

Была поправлена формулировка основания для включения в Единый реестр доменных имен:

\begin{displayquote}
	\emph{Информация о способах, методах разработки, изготовления и использования наркотических средств, психотропных веществ и их прекурсоров, новых потенциально опасных психоактивных веществ, местах их приобретения, способах и местах культивирования наркосодержащих растений. [18]}
\end{displayquote}


\section{Заключение}
С 2012 года в закон было внесено множество поправок и дополнений. Некоторые из них являются полезными, некоторые наоборот, только вредят пользователям интерннета и владельцам интернет-ресурсов.
Так, Провайдеры обязаны хранить огромное количество информации о пользователях, а также блокировать доступ к информационным ресурсам для пользователей. Но также на них ляжет непосильное финансовое и организационное бремя по воплощению закона.

В реестр запрещённой информации может попасть всё что угодно, если правильно описать любую информацию.

Таким образом, большинство из изменений, вносимых с 2012 года больше усложняют всем жизнь, чем выполняют полезную регулятивную функцию.


\end{document}

